\chapter{Podsumowanie}

Prezentowanie danych, aby było efektywne, powinno odbywać się w~skoordynowany i~zrozumiały sposób. Ich wizualizacje ułatwiają odbiorcy spojrzenie z~innej perspektywy. Pozwalają na zaobserwowanie specyficznych zjawisk, które mogą być niedostrzegalne w~surowej postaci danych. Przykładem może być tutaj radar pogodowy i~dane meteorologiczne w~ogólności. Wizualizując te dane w~kontekście wartości, ale też w~dziedzinie czasu, można dostrzec pewne wzorce. Można sklasyfikować zjawiska atmosferyczne takie jak tajfuny, czy też wyże i~niże. Można dostrzec dynamikę tych zjawisk. Takie informacje mogą następnie być wykorzystane do wytrenowania odpowiednich sieci neuronowych, aby taka klasyfikacja mogła zostać wykonana~automatycznie. 

Wizualizacje mają również wielką wagę dla komunikacji z~nieprofesjonalnym odbiorcą, którego wiedza nie pozwala przeanalizować danych wchodząc w~szczegóły danej dziedziny. Może ona być użyta, żeby zapoznać odbiorcę z~tematem, zachęcić go do jego poznania i~w sposób zrozumiały naprowadzić go do wyciągnięcia zakładanych przez twórcę wizualizacji wniosków. Wizualizacja taka musi być łatwa w~obsłudze i~nie zniechęcać odbiorcy zbyt dużą ilością prezentowanych danych.

Zaprojektowany system stara się rozwiązywać problemy i~proponować nowe podejście w~dziedzinie opisu i~wyświetlania wizualizacji. Działa on w~mobilnym i~wysoce kompatybilnym z~wieloma urządzeniami środowisku przeglądarki internetowej. Ma to swoje wady, które wynikają z~narzuconej izolacji tego środowiska wpływające na wydajność oraz wymusza specyficzną dla środowiska rozwiązania. System dostarcza dobrze udokumentowane interfejsy, służące do opisu wizualizacji i~daje możliwość wykonywania kodu podczas inicjowania obiektów sceny. Nie ogranicza to twórcy wizualizacji, który dzięki temu może projektować własne złożone podsystemy obsługujące zagadnienia z~różnych dziedzin. Wprowadzone mechanizmy działania kamery wprowadzają sposób orientacji i~niewielkie ograniczenia, specyficzne dla rozwiązywanego problemu, jakim jest wyświetlanie danych geoprzestrzennych. Wyzwaniem tutaj okazało się zaprojektowanie interfejsu obsługi zachowania kamery, który dobrze współpracuje z~obiektem~wizualizacji.

Należy wspomnieć również o~architekturze komponentu Silnika. Jego budowa oparta na frameworku Vue.js daje możliwość osadzenia go w~innym projekcie opartym na tym frameworku. Użycie języka TypeScript, stanowiący nadzbiór języka JavaScript, zmniejsza ryzyko błędów programisty, ułatwia testowanie rozwiązań oraz upraszcza dokumentację projektu. Komponent ten jest budowany do postaci biblioteki gotowej do instalacji z~użyciem narzędzia Webpack, co w~związku z~niepełną kompatybilnością ekosystemu Vue.js z~językiem TypeScript generuje problemy. W~podobnych projektach autor zaleca więc przetestowanie procesu budowania biblioteki lub aplikacji na minimalnym i~kompletnym przykładzie.

Stworzone wizualizacja w~ramach projektu są przykładem możliwości ich definicji. Komponent Silnika eksportuje wizualizacje podstawowe, jakimi są wizualizacje gwiazd i~atmosfery. Eksportuje również wizualizacje złożone, które zawierają w~swojej definicji wizualizacje podstawowe w~różnej konfiguracji. Są to między innymi wizualizacje Ziemi, Satelitów, czy danych z~radaru meteorologicznego. Będąc demonstracją możliwości systemu, jak i~samego sposobu definicji wizualizacji, nie są one w~pełni zgodne z~rzeczywistością. Nie odwzorowują one zjawisk z~pełną dokładnością oraz nie badają jej poziomu, ponieważ nie taki jest ich główny cel.

Aplikacja wyświetlająca wizualizacje, stworzona z~wykorzystaniem frametorka Vue.js, niejako spaja wszystkie elementy bazowe stworzone wcześniej. Importuje ona zbudowaną bibliotekę Silnika i~odpowiedzialna jest za wyświetlanie metadanych wizualizacji oraz ich swobodny wybór. Zagnieżdżony komponent eksportowanej biblioteki odpowiada za wyświetlanie pojedynczej wizualizacji. Ograniczenia, wynikające z~opisanego w~pracy procesu budowania biblioteki, zrodziły konieczność modyfikacji procesu budowania aplikacji. Zaszła potrzeba ręcznego przekopiowania plików zasobów wymaganych przez wizualizacje przykładowe. Dlatego też autor pracy po raz kolejny podkreśla konieczność przeanalizowania wymagań związanych z~ekosystemem, w~którym stworzona biblioteka ma działać i~przetestowania integracji wszystkich komponentów na minimalnym przykładzie.

System \texttt{GeoVis} jest dziełem prototypowym, gdzie niektóre problemy nie zostały rozwiązane. System w~przyszłości musi być przetestowany pod względem kompatybilności z~szeroką gamą urządzeń i~przeglądarek internetowych. Aplikacja powinna zostać zbudowana do wersji produkcyjnej i~udostępniona za pomocą serwera plików statycznych. Dopracowany i~rozwinięty powinien zostać również system testów wizualnej regresji generowanej grafiki przez komponent Silnika. Należy również lepiej zbadać ograniczenia dotyczące precyzji reprezentacji sceny, dla wizualizacji równocześnie małej i~dużej skali. Kolejnym krokiem rozwoju projektu może być refaktoryzacja struktury klas, związanych z~obsługą i~definicją wizualizacji, zapewniając lepszą obsługę asynchroniczności i~usprawniając zarządzanie pamięcią. W~projekcie zawsze mogą pojawić się nowe przykładowe wizualizacje, a~elementy systemów przez nie współdzielone można wyodrębnić i~stworzyć dodatkowe moduły, które mogą być użyte w~definicji pokrewnych wizualizacji. Mimo to projekt spełnia wymagania przedstawione w~rozdziale~\ref{chap:Requirements}, a~zagadnienia na jego łamach opracowane, pozwoliły pozyskać dużą wiedzę z~zakresu generowania grafiki trójwymiarowej, architektury systemów oraz aplikacji webowych.

% Koniec no nareszcie :)