\chapter{Wymagania}

Ze względu na możliwy podział funkcjonalności projektu na wiele typów, zdefiniowano następujące pojęcia:
\begin{enumerate}
    \item Silnik - zbiór komponentów odpowiedzialnych za definicję i~wyświetlenie wizualizacji.
    \item Wizualizacja - konfigurowalny widok przedstawiający obiekty, których położenie zdefiniowano za pomocą współrzędnych geograficznych, na powierzchni sfery.
    \item Aplikacja - uruchomiona w~przeglądarce użytkownika strona umożliwiająca wybór i~wyświetlenie wizualizacji.
\end{enumerate} 

Silnik dostarcza komponenty i~interfejs programistyczny, dzięki którym można definiować, wyświetlać i~zarządzać wizualizacją.
Pozwala także na zdefiniowane wielu niezależnych wizualizacji. Z tego powodu można wyróżnić dwa typy użytkowników:

\begin{enumerate}
    \item Twórcę wizualizacji,
    \item Odbiorcę wizualizacji.
\end{enumerate}

Wymagania aplikacji zostały zdefiniowane z~podziałem na typ użytkownika.
Struktura danych definiująca renderowany obraz, zwana dalej będzie sceną. Opis zachowań i~implementacji projektu w~dalszej części pracy będzie odnosił się do numeru wymagania w~celu wskazania jego spełnienia.

\newcommand{\req}[3]{
    \stepcounter{#2}
    #1\_\arabic{#2} & #3 \\
    \hline
}

\section{Twórca wizualizacji}

\subsection{Wymagania funkcjonalne}

\newcounter{c_RA}

\begin{table}[H]
    \centering
    \begin{tabularx}{\textwidth}{|l|X|}
        \hline
        Numer & Wymaganie \\
        \hline
        \hline
        \req{RA}{c_RA}{Twórca może zdefiniować metadane wizualizacji określone przez interfejs Silnika.}
        \req{RA}{c_RA}{Twórca może zdefiniować statyczną scenę określając położenie obiektów na sferze z~wykorzystaniem długości i~szerokości geograficznej.}
        \req{RA}{c_RA}{Twórca do definicji sceny może wykorzystać interfejs tworzenia obiektów dostarczony przez aplikację lub załadować obiekty, materiały i~tekstury z~zewnętrznego źródła.}
        \req{RA}{c_RA}{Twórca może zagnieżdżać sceny predefiniowane w~silniku, oraz sceny wcześniej stworzonych przez siebie.}
        \req{RA}{c_RA}{Twórca może parametryzować sceny w~celu określonej ich modyfikacji w~procesie zagnieżdżania.}
        \req{RA}{c_RA}{Twórca może określić parametry początkowe obserwatora, dynamikę i~zakres jego ruchów:
            \begin{enumerate}
                \item położenie,
                \item prędkość i~przyspieszenie ruchu,
                \item ograniczenie przybliżenia,
                \item ograniczenie pozycji.
            \end{enumerate}
        }
        \req{RA}{c_RA}{Twórca może zdefiniować wygląd i~funkcjonalność panelu kontrolnego. Panel ten służyć będzie do zmiany parametrów wizualizacji i~obsługiwany będzie przez odbiorcę.}
        \req{RA}{c_RA}{Twórca, poprzez interfejs programistyczny dostarczony przez silnik, może aktualizować scenę w~dowolnym momencie, określonym przez siebie w~definicji wizualizacji.}
        \req{RA}{c_RA}{Twórca może definiować zachowania, które będą odpowiedzią na zdarzenia związane z~poruszaniem się po scenie generowane przez odbiorcę.}
    \end{tabularx}
    \caption{Wymagania funkcjonalne zdefiniowane dla twórcy wizualizacji }
    \label{tab:req_author_f}
\end{table}

\subsection{Wymagania niefunkcjonalne}

\begin{table}[H]
    \centering
    \begin{tabularx}{\textwidth}{|l|X|}
        \hline
        Numer & Wymaganie \\
        \hline
        \hline
        \req{RA}{c_RA}{Silnik powinien definiować i~w sposób jasny przekazywać potencjalnemu twórcy akceptowalną strukturę danych, plików i~katalogów, określającą jedną wizualizację.}
        \req{RA}{c_RA}{Włączenie zdefiniowanej wizualizacji do ich zbioru w~aplikacji powinno ustanowione być tylko w~jednym miejscu poprzez prosty interfejs.}
        \req{RA}{c_RA}{Dane wizualizacji muszą być ładowane asynchronicznie. Dane źródłowe definiujące scenę mogą być przetwarzane po stronie odbiorcy lub być przetworzone wcześniej i~pobrane.}
    \end{tabularx}
    \caption{Wymagania niefunkcjonalne zdefiniowane dla twórcy wizualizacji }
    \label{tab:req_author_nf}
\end{table}

\section{Odbiorca wizualizacji}
\newcounter{c_RU}

\subsection{Wymagania funkcjonalne}

\begin{table}[H]
    \centering
    \begin{tabularx}{\textwidth}{|l|X|}
        \hline
        Numer & Wymaganie \\
        \hline
        \hline
        \req{RU}{c_RU}{Odbiorca może zobaczyć dane dostępnych wizualizacji.}
        \req{RU}{c_RU}{Odbiorca może wyświetlić wybraną wizualizację.}
        \req{RU}{c_RU}{Odbiorca może poruszać się po wizualizacji, zmieniając położenia kamery, używając myszki lub klawiatury.}
        \req{RU}{c_RU}{Odbiorca może zobaczyć orientację kamery relatywnie do kierunku północnego i~ją zresetować.}
        \req{RU}{c_RU}{Odbiorca może wyświetlić lub ukryć panel sterujący wizualizacją dostarczony przez twórcę.}
    \end{tabularx}
    \caption{Wymagania funkcjonalne zdefiniowane dla odbiorcy wizualizacji }
    \label{tab:req_user_f}
\end{table}

\subsection{Wymagania niefunkcjonalne}

\begin{table}[H]
    \centering
    \begin{tabularx}{\textwidth}{|l|X|}
        \hline
        Numer & Wymaganie \\
        \hline
        \hline
        \req{RU}{c_RU}{Każda akcja użytkownika związana ze sterowaniem kamerą może zostać wykonana używając myszki lub równolegle klawiatury.}
       
    \end{tabularx}
    \caption{Wymagania funkcjonalne zdefiniowane dla odbiorcy wizualizacji }
    \label{tab:req_user_nf}
\end{table}

\section{Aplikacja}

\subsection{Wymagania niefunkcjonalne}

\begin{table}[H]
    \centering
    \begin{tabularx}{\textwidth}{|l|X|}
        \hline
        Numer & Wymaganie \\
        \hline
        \hline
        \req{RU}{c_RU}{Aplikacja powinna być stroną typu \textit{Single Page Application}.}
        \req{RU}{c_RU}{Jeśli to możliwe aplikacja powinna wykorzystywać sprzętową akcelerację obliczeń graficznych.}
        \req{RU}{c_RU}{Aplikacja powinna ustawiać i~obsługiwać adres URL w~przeglądarce definiujący wyświetlaną wizualizację.}
       
    \end{tabularx}
    \caption{Wymagania funkcjonalne zdefiniowane dla aplikacji }
    \label{tab:req_user_nf}
\end{table}



